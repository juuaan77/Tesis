
\documentclass{beamer}
%
% Choose how your presentation looks.
%
% For more themes, color themes and font themes, see:
% http://deic.uab.es/~iblanes/beamer_gallery/index_by_theme.html
%
\graphicspath{{./figuras/}}
\mode<presentation>
{
    \usebackgroundtemplate{\includegraphics[width=\paperwidth]{tesis.jpg}}
  \usetheme{default}      % or try Darmstadt, Madrid, Warsaw, ...
  \usecolortheme{default} % or try albatross, beaver, crane, ...
  \usefonttheme{default}  % or try serif, structurebold, ...
  \setbeamertemplate{navigation symbols}{}
  \setbeamertemplate{caption}[numbered]
} 

\usepackage[english]{babel}
\usepackage[utf8x]{inputenc}

\title[Your Short Title]{Infraestructura tecnológica virtual con automatización y orquestación.}
\author{Arese, Juan Pablo - Diers, Werner Christian}
\institute{Facultad de Ciencias Exactas, Físicas y Naturales - UNC}
\date{Diciembre 2016}

\begin{document}

\begin{frame}
  \titlepage
\end{frame}

\section{Organización de la Presentación}

\begin{frame}{Organización de la Presentación}

\vspace{-1.5cm}
La precentación esta organizada de manera incremental explicando cada uno de los siguientes puntos.\\
\begin{itemize}
    \item Introduccion.
    \item Objetivos.
    \item Arquitectura.
    \item Desarrollo del sistema.
    \begin{itemize}
        \item Herramienta de virtualizacion.
        \item Herramienta de aprovicionamiento.
        \item Herramienta de orquestacion.
        \item Interfaz grafica.
    \end{itemize}
\end{itemize}

\end{frame}


\section{Introduccion}


\begin{frame}
    %\vspace{}
    \Huge
    \centering
    \textbf{ Introduccion }

\end{frame}


\begin{frame}{Introduccion}

\vspace{-1.5cm}
La cantidad de servicios y servidores necesarios en las organizaciones tiende a ser cada vez mayor.
Cada día los despliegues son más complejos, siendo necesario trabajar con aplicaciones clusterizadas, múltiples datacenters,etc.
Este tipo de tareas ya no pueden realizarse individualmente para cada máquina y menos aún en entornos escalables donde pueden haber cientos de nodos.

\begin{block}{}
Hablar de orquestación implica eficiencia en los recursos, tanto humanos como computacionales, y por ello implica hablar de virtualización, provisionamiento y datacenters dinámicos.
\end{block}

%\begin{itemize}
%  \item Your introduction goes here!
%  \item Use \texttt{itemize} to organize your main points.
%\end{itemize}

%\vskip 1cm

%\begin{block}{Examples}
%Some examples of commonly used commands and features are included, to help you get started.
%\end{block}

\end{frame}

\section{Objetivos}


\begin{frame}
    %\vspace{}
    \Huge
    \centering
    \textbf{ Objetivos }

\end{frame}

\begin{frame}{Objetivo principal}

\vspace{-1.5cm}
Desarrollar un sistema que integre diferentes tecnologías para implementar las técnicas de orquestación, virtualización y configuración automática para facilitar la gestión de servidores virtuales y sus servicios asociados.

\end{frame}


\begin{frame}{Objetivos secundarios}

\vspace{-1.5cm}
\begin{itemize}
    \item Instalar y utilizar sistemas operativos para servidor.
    \item Emplear herramientas de virtualización.
    \item Usar herramientas de aprovisionamiento.
    \item Utilizar herramientas de orquestación.
    \item Analizar protocolos para booteo a través de la red.
\end{itemize}

\end{frame}

\section{Arquitectura}

\begin{frame}
    %\vspace{}
    \Huge
    \centering
    \textbf{ Arquitectura }

\end{frame}

\begin{frame}{Arquitectura}

\vspace{-1.5cm}
La arquitectura seleccionada es la de Cliente-Servidor
Las tareas del servidor son las siguientes:
\begin{itemize}
    \item Crear las maquinas virtuales (En caso de utilizar un servidor virtual).
    \item Asignar direcciones IP via servidor DHCP.
    \item Aprovicionar la maquina con el SO deseado y los parametros de configuracion establecidos.
    \item Orquestar las politicas definidas para la maquina.
\end{itemize}
 INSERTAR FIGURA ACA
%\begin{figure}[ht]
%    \centering
%    \vspace{-0.50cm}
%    \scalebox{0.25}{\includegraphics[angle=0]{./oh_source.eps}}
%\end{figure}

\end{frame}

\section{Desarrollo}

\begin{frame}
    %\vspace{}
    \Huge
    \centering
    \textbf{ Desarrollo }

\end{frame}

\begin{frame}{Desarrollo - Virtualizacion}

\vspace{-1.5cm}
La herramienta de virtualizacion seleccionada es Qemu/KVM.\\
INSERTAR ACA EL TIPO DE VIRTUALIZACION DE KVM
INSERTAR ACA LA IMAGEN DE LA ARQUITECTURA DE LA VIRTUALIZACION DE KVM
\end{frame}



\end{document}

              